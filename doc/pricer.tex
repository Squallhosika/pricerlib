%% Based on a TeXnicCenter-Template by Tino Weinkauf.
%%%%%%%%%%%%%%%%%%%%%%%%%%%%%%%%%%%%%%%%%%%%%%%%%%%%%%%%%%%%%

%%%%%%%%%%%%%%%%%%%%%%%%%%%%%%%%%%%%%%%%%%%%%%%%%%%%%%%%%%%%%
%% HEADER
%%%%%%%%%%%%%%%%%%%%%%%%%%%%%%%%%%%%%%%%%%%%%%%%%%%%%%%%%%%%%
%%\documentclass[a4paper,twoside,10pt]{report}


\documentclass[11pt,twoside,a4paper]{article}

% Alternative Options:
%	Paper Size: a4paper / a5paper / b5paper / letterpaper / legalpaper / executivepaper
% Duplex: oneside / twoside
% Base Font Size: 10pt / 11pt / 12pt


%% Language %%%%%%%%%%%%%%%%%%%%%%%%%%%%%%%%%%%%%%%%%%%%%%%%%
\usepackage[USenglish]{babel} %francais, polish, spanish, ...
\usepackage[T1]{fontenc}
\usepackage[ansinew]{inputenc}

\usepackage{lmodern} %Type1-font for non-english texts and characters


%% Packages for Graphics & Figures %%%%%%%%%%%%%%%%%%%%%%%%%%
\usepackage{graphicx} %%For loading graphic files
%\usepackage{subfig} %%Subfigures inside a figure
%\usepackage{pst-all} %%PSTricks - not useable with pdfLaTeX

%% Please note:
%% Images can be included using \includegraphics{Dateiname}
%% resp. using the dialog in the Insert menu.
%% 
%% The mode "LaTeX => PDF" allows the following formats:
%%   .jpg  .png  .pdf  .mps
%% 
%% The modes "LaTeX => DVI", "LaTeX => PS" und "LaTeX => PS => PDF"
%% allow the following formats:
%%   .eps  .ps  .bmp  .pict  .pntg


%% Math Packages %%%%%%%%%%%%%%%%%%%%%%%%%%%%%%%%%%%%%%%%%%%%
\usepackage{amsmath}
\usepackage{amsthm}
\usepackage{amsfonts}


%% Line Spacing %%%%%%%%%%%%%%%%%%%%%%%%%%%%%%%%%%%%%%%%%%%%%
%\usepackage{setspace}
%\singlespacing        %% 1-spacing (default)
%\onehalfspacing       %% 1,5-spacing
%\doublespacing        %% 2-spacing


%% Other Packages %%%%%%%%%%%%%%%%%%%%%%%%%%%%%%%%%%%%%%%%%%%
%\usepackage{a4wide} %%Smaller margins = more text per page.
%\usepackage{fancyhdr} %%Fancy headings
%\usepackage{longtable} %%For tables, that exceed one page


%%%%%%%%%%%%%%%%%%%%%%%%%%%%%%%%%%%%%%%%%%%%%%%%%%%%%%%%%%%%%
%% Remarks
%%%%%%%%%%%%%%%%%%%%%%%%%%%%%%%%%%%%%%%%%%%%%%%%%%%%%%%%%%%%%
%
% TODO:
% 1. Edit the used packages and their options (see above).
% 2. If you want, add a BibTeX-File to the project
%    (e.g., 'literature.bib').
% 3. Happy TeXing!
%
%%%%%%%%%%%%%%%%%%%%%%%%%%%%%%%%%%%%%%%%%%%%%%%%%%%%%%%%%%%%%

%%%%%%%%%%%%%%%%%%%%%%%%%%%%%%%%%%%%%%%%%%%%%%%%%%%%%%%%%%%%%
%% Options / Modifications
%%%%%%%%%%%%%%%%%%%%%%%%%%%%%%%%%%%%%%%%%%%%%%%%%%%%%%%%%%%%%

%\input{options} %You need a file 'options.tex' for this
%% ==> TeXnicCenter supplies some possible option files
%% ==> with its templates (File | New from Template...).



%%%%%%%%%%%%%%%%%%%%%%%%%%%%%%%%%%%%%%%%%%%%%%%%%%%%%%%%%%%%%
%% DOCUMENT
%%%%%%%%%%%%%%%%%%%%%%%%%%%%%%%%%%%%%%%%%%%%%%%%%%%%%%%%%%%%%
\begin{document}

\pagestyle{empty} %No headings for the first pages.


%% Title Page %%%%%%%%%%%%%%%%%%%%%%%%%%%%%%%%%%%%%%%%%%%%%%%
%% ==> Write your text here or include other files.

%% The simple version:
\title{Pricer MC PDE}
\author{Jonathan Lerch}
%\date{} %%If commented, the current date is used.
\maketitle

%% The nice version:
%\input{titlepage} %%You need a file 'titlepage.tex' for this.
%% ==> TeXnicCenter supplies a possible titlepage file
%% ==> with its templates (File | New from Template...).


%% Inhaltsverzeichnis %%%%%%%%%%%%%%%%%%%%%%%%%%%%%%%%%%%%%%%
%%\tableofcontents %Table of contents
%%\cleardoublepage %The first chapter should start on an odd page.

\pagestyle{plain} %Now display headings: headings / fancy / ...



%% Chapters %%%%%%%%%%%%%%%%%%%%%%%%%%%%%%%%%%%%%%%%%%%%%%%%%
%% ==> Write your text here or include other files.

%\input{intro} %You need a file 'intro.tex' for this.


%%%%%%%%%%%%%%%%%%%%%%%%%%%%%%%%%%%%%%%%%%%%%%%%%%%%%%%%%%%%%
%% ==> Some hints are following:

%%\chapter{Some small hints}\label{hints}

\section{The SDE}\label{umlauts}
We will do a pricer by MC and PDE on the following SDE:

\begin{eqnarray*}
  dS_t & = & \sigma(S_t) dW_t\\
	S_0  &= & s0
\end{eqnarray*}


\section{MC Engine}

For the MC engine we used two schemes, the Euler :

\begin{eqnarray*}
  \tilde{S}_{t_{k+1}} & = & \tilde{S}_{t_{k}} +  \sigma(\tilde{S}_{t_{k}}) \left( W_{t_{k+1}} - W_{t_{k}} \right) \\
	\tilde{S}_{t_{0}}   & = & s0
\end{eqnarray*}

and the Milstein:

\begin{eqnarray*}
  \tilde{S}_{t_{k+1}} & = & \tilde{S}_{t_{k}} +  \sigma(\tilde{S}_{t_{k}}) \left( W_{t_{k+1}} - W_{t_{k}} \right) \\
	& + & \frac{1}{2} \sigma(\tilde{S}_{t_{k}}) \sigma'(\tilde{S}_{t_{k}}) \left( \left( W_{t_{k+1}} - W_{t_{k}} \right)^2 - \left( t_{k+1} - t_{k}\right) \right)\\
	\tilde{S}_{t_{0}}   & = & s0
\end{eqnarray*}

Because of its stronger rate of convergence we mainly used the Milstein.

\section{PDE Engine}

From the SDE we can deduce the following backward PDE for a final payoff $h$:

\begin{eqnarray*}
  \frac{\partial V}{\partial t} + \frac{1}{2} \sigma^2(x) \frac{\partial^2 V}{\partial x^2} = 0\\
	V(x,T) = h(x)
\end{eqnarray*}

We now descretize this PDE with an explicit scheme:

\begin{eqnarray*}
  \frac{V^k_t - V^k_{t-1}}{\Delta t(t-1,t) } + \frac{1}{2} \sigma^2(x^k_t) \frac{V^{k+1}_t + V^{k-1}_{t} - 2 V^{k}_{t} }{h^2} = 0\\
	V^k_T = h(x^k_T)
\end{eqnarray*}

an implicit scheme:

\begin{eqnarray*}
  \frac{V^k_t - V^k_{t-1}}{\Delta t(t-1,t) } + \frac{1}{2} \sigma^2(x^k_{t-1}) \frac{V^{k+1}_{t-1} + V^{k-1}_{t-1} - 2 V^{k}_{t-1} }{h^2} = 0\\
	V^k_T = h(x^k_T)
\end{eqnarray*}

and a theta-scheme mix of explicit and implicit:

\begin{eqnarray*}
  \frac{V^k_t - V^k_{t-1}}{\Delta t(t-1,t) } +  \frac{1}{2} \theta \sigma^2(x^k_t) \frac{V^{k+1}_t + V^{k-1}_{t} - 2 V^{k}_{t} }{h^2}\\
	+ \frac{1}{2}  (1 - \theta) \sigma^2(x^k_{t-1}) \frac{V^{k+1}_{t-1} + V^{k-1}_{t-1} - 2 V^{k}_{t-1} }{h^2} = 0\\
	V^k_T = h(x^k_T)
\end{eqnarray*}

In our implementation the discretization in time is not constant like suggest by the $\Delta t(t-1,t)$. But the discretization in underlying is.

For the boundary condition we have impose :

\begin{eqnarray*}
  \frac{\partial^2 V}{\partial x^2} = 0
\end{eqnarray*}

A reasonable assumption for most of the product. This result in having $V^k_{t-1} = V^k_t$ for the two boundary in $k$.

For the choice of $k_{max}$ and $k_{min}$, I express my range in term of standard deviation of my process. This method have to by improve. In particular, the way I evaluate the standard deviation of my process (I used only the volatility at a specific point).

Because the implicit scheme always converge (even if it is at a lower rate), we mainly use this one.

\section{Remark on the code}

I used boost for the random number generator and the distribution.
I used the numerical recipes for the inversion of the tridiagonal matrix and a basic interpolator.
%% <== End of hints
%%%%%%%%%%%%%%%%%%%%%%%%%%%%%%%%%%%%%%%%%%%%%%%%%%%%%%%%%%%%%



%%%%%%%%%%%%%%%%%%%%%%%%%%%%%%%%%%%%%%%%%%%%%%%%%%%%%%%%%%%%%
%% BIBLIOGRAPHY AND OTHER LISTS
%%%%%%%%%%%%%%%%%%%%%%%%%%%%%%%%%%%%%%%%%%%%%%%%%%%%%%%%%%%%%
%% A small distance to the other stuff in the table of contents (toc)
\addtocontents{toc}{\protect\vspace*{\baselineskip}}

%% The Bibliography
%% ==> You need a file 'literature.bib' for this.
%% ==> You need to run BibTeX for this (Project | Properties... | Uses BibTeX)
%\addcontentsline{toc}{chapter}{Bibliography} %'Bibliography' into toc
%\nocite{*} %Even non-cited BibTeX-Entries will be shown.
%\bibliographystyle{alpha} %Style of Bibliography: plain / apalike / amsalpha / ...
%\bibliography{literature} %You need a file 'literature.bib' for this.

%% The List of Figures
%%\clearpage
%%\addcontentsline{toc}{chapter}{List of Figures}
%%\listoffigures

%% The List of Tables
%%\clearpage
%%\addcontentsline{toc}{chapter}{List of Tables}
%%\listoftables


%%%%%%%%%%%%%%%%%%%%%%%%%%%%%%%%%%%%%%%%%%%%%%%%%%%%%%%%%%%%%
%% APPENDICES
%%%%%%%%%%%%%%%%%%%%%%%%%%%%%%%%%%%%%%%%%%%%%%%%%%%%%%%%%%%%%
\appendix
%% ==> Write your text here or include other files.

%\input{FileName} %You need a file 'FileName.tex' for this.


\end{document}

